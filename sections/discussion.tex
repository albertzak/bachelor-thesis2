\cleardoublepage
\section{Discussion}

The following two sections qualitatively review the contribution with respect to the stated goals, and expose limitations of the current implementation as well as discuss flaws inherent to the design.

\subsection{Advantages}

\paragraph{One time setup.} The tool is meant to be used as part of a \acrfull{ci} pipeline. Invoking a single command generates a deployable artifact from a checkout of the code base. Version identifiers are constructed out of commit timestamps and hashes, and \acrfull{appup} are generated on a best-effort basis. Previous releases are fetched from a central store, where the newly generated release artifact is uploaded to.

\paragraph{.} The \acrshort{cli} build tool is installed with a single command that does not need super user permissions, is not dependent on any specific package manager, and has no dependencies except \lstinline|curl|, \lstinline|bash| and Git––which can safely be assumed to exist on the system.

\paragraph{\acrshort{ci} support.} The tool has been shown to work on at least six common hosted \acrfull{ci} providers: \emph{Travis \acrshort{ci}, Codeship, Circle \acrshort{ci}, Wercker, Semaphore,} and \emph{Shippable}. Build run time and setup effort are comparable across providers.

\paragraph{Declarative.} The command to start a build is always the same, there are no flags or parameters. All settings such as authentification credentials for the store, or the version of Erlang/Elixir to use are passed to the tool via environment variables. The build process requires no interaction.

\paragraph{Single cache.}  Instead of having to configure and maintain multiple cache paths that may change in the future, the cache locations of various tools used within the build process are consolidated inside one directory.

\paragraph{Normalized environment.} The tool detects whether it is being run directly on a \acrshort{vm}, in which case it attempts to normalize the environment by starting a container. Nevertheless, the tool aims to be well behaved with respect to its environment.

\paragraph{Read only.} The tool never modifies the original project folder. All operations happen on temporary copies, preferably on a \acrshort{ram} disk, and the only artifact produced by a build run is a single deployable tarball. The compressed release artifact is either directly uploaded to a remote store, or written to the file system as specified by an environment variable.

% --
\cleardoublepage
\subsection{Limitations}

\paragraph{Installer.}\label{sec:curlpipesh} Distributing the \acrshort{cli} tool by downloading a script and piping it to the shell, thereby immediately executing its arbitrary contents may appear to be unsafe. Yet there is currently no simpler solution. The content delivery network that serves the installer script is configured to only accept connections over \acrshort{http}S, and \lstinline|curl| verifies the validity of the certificate. The installer is wrapped in a shell function to guard against executing a partially downloaded script. Besides, those concerned about the security implications can circumvent the installer by manually cloning a vetted checkout of the \acrshort{cli} repository.

\paragraph{Linux only.} Since the build process always runs inside a Linux container, the resulting release artifact only runs on Linux systems as well. Note that the \acrshort{cli} tool itself may be invoked on other \acrshort{os}s where Docker is supported, such as macOS, as the Docker client will then transparently run the container on a virtualized Linux kernel.~\cite{docker:docs} Consequently, the built release still only supports Linux.

\paragraph{Minimal customization.} The tool aims to cover only the most common default project configurations. This implies that all configuration files for the respective tools are in their standard locations, and that the project is set up like the scaffolding generated by the \lstinline|beamup new| command. Advanced features of the build tools such as build profiles, native dependencies, or multiple releases are not supported.

\paragraph{Already containerized pipeline.} All but one of the tested \acrshort{ci} providers run the build inside a preconfigured container, with only Travis \acrshort{ci} offering the choice between a container or a \acrshort{vm}. When the container has already been started, the launcher cannot make certain optimizations. Namely, it cannot mount a \acrshort{tmpfs} \acrshort{ram} disk, or consolidate the cache folders. The tool must also assume that the correct version of Erlang/\acrshort{otp} is already installed.

\paragraph{High \acrshort{ram} or disk usage during build.}
To avoid disk thrashing and to improve performance, the tool attempts to copy the project folder onto a \acrshort{ram} scratch disk. This working copy is naively duplicated multiple times during the build process, at least twice for every upgrade from a previous release. Note that in case the \acrshort{ram} disk fills the available memory, the \acrshort{os} falls back to using swap space. Nevertheless, the builder attempts to keep the number of concurrently existing duplicates to a minimum.


\paragraph{Store is an active component.} The builder currently supports two backends for storing built artifacts. The release tarball may either be output to the local file system, or uploaded to the store, which is an active server component that must be deployed and maintained. The current feature set implies that an active store component is unnecessary, and a backend for e.g.~a simple object storage would suffice.

The \acrshort{api} and implementation of the store server is incomplete and will in the future be extended to cover additional aspects of release handling, deployment and bootstrapping of nodes.

\paragraph{Possibly unnecessary downloading of previous releases.} The tool currently downloads the entirety of all previous releases to generate upgrade instructions. An alternative would be to locally check out the previous commits to upgrade from, compile each previous release from the checked out code, and then to compare them to generate upgrade instructions, as documented by~\cite{rebar3appup}. This paper has not evaluated advantages or limitations of a local-only approach.

\paragraph{Releases contain debug information.} Expanding the point made above, the current approach might be flawed in the following way: To generate upgrade instructions from previous releases, the bytecode files needed to be compiled with included debug information. This can be used to reconstruct the Erlang source code.~\cite{doc:otp}

\paragraph{Transient network failures.} Out of 448 test runs, three failed. Two failed attempts to download a dependency could have been caught and retried by the build tool. Retrying on network failures should arguably be a concern of the lower-level tool which initiated the request, yet a similar case could be made for handling them in the proposed tool. Unstable builds interfere with the \acrshort{ci} workflow, thus only irrecoverable errors should bubble up to affect the \acrshort{ci} build status.

\paragraph{Machine architecture support.} Though untested, the tool should be able to run on various machine architectures, and produce releases runnable on all of the architectures for which official Erlang/OTP Docker images~\cite{docker:erlang} are available: \lstinline|amd64|, \lstinline|arm32v7|, \lstinline|arm64v8|, \lstinline|i386|, \lstinline|ppc64le|, \lstinline|s390x|. However, the tool has not been tested on any machine architectures except \lstinline|amd64|. There is also no cross compilation support, nor is it expected that dependencies in the form of Erlang \acrfull{nif}~\cite{doc:otp} can be part of projects built with the proposed tool.

\paragraph{Best-effort \acrshort{appup} generation.} The build tool trades full control over the build process for hands-off operation. In certain situations with complicated dependencies between modules, processes, or even nodes the algorithm by~\cite{rebar3appup}
for generating \acrfull{appup} files may produce incorrect results, leading to failed \acrshort{dsu} attempts. However, if such situations are known beforehand, e.g.~through careful testing of upgrades, the developer may add handwritten \acrshort{appup} template files (\lstinline|*.appup.src|) to the respective applications to guide the algorithm.

\paragraph{No eviction of old releases.} The current implementation of the build tool never deletes old artifacts from the store. It also always fetches all previously stored releases to build upgrades from, leading to increasingly long build times.
