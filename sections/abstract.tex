\cleardoublepage{}

\section*{Abstract}

Highly available Erlang/\acrshort{otp} systems wishing to take advantage of \acrfull{dsu} require a manual deployment process that is at odds with the practice of \acrlong{cd}.

Previous work on automating Erlang/OTP release handling relied on push-based, imperative semantics

deployment with imperative semantics.

odern tooling expecting immutable containers.

This thesis presents a way to bootstrap nodes, requiring only minimal configuration,

pull-based node bootstrapping

revious work on automating Erlang/\acrshort{otp} Release generation has addressed parts of the process by abstracting the low-level mechanics of the build step, but tasks such as versioning and handling artifacts were left to the developer.

This thesis presents a release generation tool designed for hands-off operation as part of a Continuous Integration pipeline.

Tight coupling with Git allows reliable identification of releases by discarding numeric versions in favor of commit hashes, and a centralized release store takes care of handling artifacts.

Evaluation of a reference implementation on six hosted Continuous Integration providers (\emph{CircleCI, Codeship, Semaphore, Shippable, Travis CI,} and \emph{Wercker}) demonstrates comparable run time and ease of setup.
